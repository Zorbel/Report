\newpage
\chapter*{Conclusions}
\addcontentsline{toc}{chapter}{\numberline{}Conclusions}

Developing a project like DemoCritics has not been simple. Being an application \textbf{made from scratch} from our own idea and initiative, it requires a lot of dedication in the form of time and motivation. Moreover, if we talk about a topic such as \textbf{politics}, which unfortunately doesn't seem to attract all the attention and interest of the people it should really have as we are part of our \textbf{society}. In this sense, the time spent on the previous work of \textbf{planning, research, design and evaluation} of prototypes was necessary to establish a solid foundation on which to build the implementation of \textbf{useful social characteristics}.

This process demonstrates the importance of the developer, usually used to have a fairly low vision of \textbf{technical} aspects, being aware of the \textbf{need to involve in developing other voices}. These voices may not have the expertise needed to '' implement '' the idea, but it does provide a broader overview that helps to better understand the real needs of the user, which ultimately is the recipient of the application. Therefore, \textbf{conducting interviews and subsequent evaluations} to take into account the opinion of the potential '' client '' has confronted us with a necessary reality for any developer who wants to get a serious product. Moreover when \textbf{we consider DemoCritics as an application that goes beyond the scope of the Final Degree Project and will become a useful tool in the near future}.

The results achieved are reflected in the current state of DemoCritics. For the first time citizens have within reach a \textbf{space} to consult \textbf{all political options} (wherever they come), learn about them, discuss them and value them. As well as participate in a \textbf{participatory process} of citizens' proposals, giving them the opportunity to give visibility to their proposal or collaborate in the development of others. The social potential of the application is also utilised by the possibility of review, comment and bookmark all the contents of the application.

Also it makes \textbf{real-time collaboration technology} available for the community of \textbf{Free Software} developers in Android, so they could use of these features in future development of applications. As well as the application that contains the Free Software platform for \textbf{citizen participation} DemoCritics, that any user could \textbf{adapt to the needs} and options of their region, country or community.

There is still \textbf{room for improvements}, because this platform is a first version of a project that, with \textbf{further research and development}, we intend to put into practice in a real scenario of the \textbf{upcoming general elections} that will be held during the coming months.


\chapter*{Conclusiones}
\addcontentsline{toc}{chapter}{\numberline{}Conclusiones}

Desarrollar un proyecto como DemoCritics no ha sido tarea sencilla. Al tratarse de una aplicación que \textbf{parte de cero} desde nuestra propia idea e iniciativa, exige bastante dedicación en forma de tiempo y motivación. Más aún si hablamos de una temática como es la \textbf{política}, que desgraciadamente parece no atraer toda la atención e interés de las personas que realmente debería tener como actividad de la que formamos parte como \textbf{sociedad}. En este sentido el tiempo dedicado a las labores previas de \textbf{planificación, investigación, diseño y evaluación} de prototipos era necesario para establecer unas bases sólidas sobre las que asentar la implementación de características que resultaran \textbf{de utilidad social}.

Este proceso demuestra la importancia de que el desarrollador, acostumbrado normalmente a tener una visión bastante reducida a aspectos \textbf{técnicos}, sea consciente de la \textbf{necesidad de implicar en el desarrollo otras voces}. Dichas voces quizás no posean los conocimientos técnicos necesarios para ''implementar'' la idea, pero sí que aportan una visión de conjunto más amplia que ayuda a entender mejor las necesidades reales del usuario, que a fin de cuentas es el destinatario de la aplicación. Realizar por tanto \textbf{entrevistas y posteriores evaluaciones} para tener en cuenta la opinión del futurible ''cliente'' nos ha enfrentado a una realidad necesaria para cualquier desarrollador que quiera conseguir un producto serio. Más aun cuando \textbf{nos planteamos DemoCritics como una aplicación que trascienda el alcance del Trabajo de Fin de Grado y se convierta en una herramienta útil en el futuro próximo.}

Los resultados obtenidos se plasman en el estado actual de DemoCritics. Por vez primera la ciudadanía tiene al alcance de sus manos un \textbf{espacio} donde consultar \textbf{todas las opciones políticas} (vengan de donde vengan), aprender sobre ellas, debatirlas o valorarlas. Así como también participar en un \textbf{proceso participativo} de propuestas ciudadanas, dándole la posibilidad de dar visibilidad a su propuesta o colaborar en el desarollo de otras. Se explota también el potencial social de la aplicación mediante la posibilidad de opinar, realizar comentarios y marcar como favorito en todo el contenido de la aplicación.

Se pone así tambien la \textbf{tecnología de colaboración en tiempo real} a disposición de la comunidad de desarrolladores de \textbf{Software libre} en Android para que hagan uso de estas funcionalidades en sus futuras aplicaciones. Así como también la aplicación que contiene la plataforma de Software libre de \textbf{participación ciudadana} de DemoCritics, que cualquier usuario podría \textbf{adaptar a las necesidades} y opciones de su región, país o comunidad.

Existe todavía \textbf{margen de mejora}, pues la presente plataforma es una primera versión de un proyecto en el que pretendemos \textbf{seguir investigando y desarrollando} para ponerla en práctica en un escenario real de cara a las \textbf{próximas elecciones generales} que se celebrarán durante los próximos meses.


