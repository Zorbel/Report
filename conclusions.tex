\newpage
\thispagestyle{sectioned}
\chapter{Conclusiones}

Desarrollar un proyecto como DemoCritics no es una tarea baladí. Al tratarse de una aplicación que parte de cero desde nuestra propia idea e iniciativa, exige bastante dedicación en forma de tiempo y motivación. Más aún si hablamos de una temática como es la política, que desgraciadamente parece no atraer toda la atención e interés de las personas que relamente debería tener como actividad de la que formamos parte como sociedad. En este sentido el tiempo dedicado a las labores previas de planificación, investigación, diseño y evaluación de prototipos era necesario para establecer unas bases sólidas sobre las que asentar la futura implementación de características que resultaran a la vez útiles y atractivas para el usuario.

Este proceso fue bastante productivo y nos sirvió para depurar y pulir aspectos previos de nuestra idea inicial. Fue también enriquecedor para nosotros como experiencia, pues no hizo sino demostrarnos la importancia de que el desarrollador, acostumbrado normalmente a tener una visión   bastante reducida a aspectos técnicos, sea consciente de la necesidad de implicar en el desarrollo otras voces. Dichas voces quizás no posean los conocimientos técnicos necesarios para ''implementar'' la idea, pero sí que aportan una visión de conjunto más amplia que ayuda a entender mejor las necesidades reales del usuario, que a fin de cuentas es el destinatario de la aplicación. Realizar por tanto entrevistas y posteriores evaluaciones para tener en cuenta la opinión del futurible ''cliente'' nos ha enfrentado a una realidad necesaria para cualquier desarrollador que quiera conseguir un producto serio. Más aun cuando nos planteamos DemoCritics como una aplicación que trascendiera el alcance del Trabajo de Fin de Grado y se convertiera en una herramienta útil en el futuro.

DemoCritics ha supuesto también investigar el uso de tecnologías que no conocíamos, con el consiguiente proceso de aprendizaje que ello conlleva. Hemos tenido que enfrentarnos al estudio del funcionamiento de una plataforma ya establecida como Wave/SwellRT para llevar a cabo una migración a Android. Hemos trabajado el uso de tecnologías del lado del cliente como las conexiones desde Android y del servidor como la implementación del Service REST. Pero sobre todo, hemos aprendido a llevar una metodología de trabajo basada en el control de versiones mediante Git y GitHub.

Los resultados obtenidos se plasman en el estado actual de DemoCritics: una aplicación capaz de juntar en un solo sitio la navegación por las secciones de programas políticos y crear propuestas, con la destacada opción de hacer esto úlitmo entre varias personas de forma colaborativa y en tiempo real utilizando la tecnología de SwellRT en Android. Explotamos también el potencial social de la aplicación mediante la posibilidad de opinar, realizar comentarios y marcar como favorito, en todo el contenido de la aplicación.

\textbf{Se trata de un proyecto en el que aportamos una solución de software libre  que no existía previamente, pues las opciones de colaboración en tiempo real en Android pasaban únicamente por soluciones privativas como el Real Time API de Google}. Su código está disponible en GitHub y que cualquiera puede estudiar, contribuir, copiar y utilizar libremente. Tanto en el caso de la migración de SwellRT a Android como en el de la plataforma DemoCritics (la App, la base de datos y el Service REST). 

Queda aún trabajo por hacer, pues la presente plataforma es una primera versión de un proyecto en el que pretendemos seguir investigando y desarrollando más funcionalidades, tal y como se discutirá a continuación en el Trabajo Futuro. 

