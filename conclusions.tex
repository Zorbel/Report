\newpage
\chapter*{Conclusions}
\addcontentsline{toc}{chapter}{\numberline{}Conclusions}



\chapter*{Conclusiones}
\addcontentsline{toc}{chapter}{\numberline{}Conclusiones}

Desarrollar un proyecto como DemoCritics no ha sido tarea sencilla. Al tratarse de una aplicación que \textbf{parte de cero} desde nuestra propia idea e iniciativa, exige bastante dedicación en forma de tiempo y motivación. Más aún si hablamos de una temática como es la \textbf{política}, que desgraciadamente parece no atraer toda la atención e interés de las personas que realmente debería tener como actividad de la que formamos parte como \textbf{sociedad}. En este sentido el tiempo dedicado a las labores previas de \textbf{planificación, investigación, diseño y evaluación} de prototipos era necesario para establecer unas bases sólidas sobre las que asentar la implementación de características que resultaran \textbf{de utilidad social}.

Este proceso demuestra la importancia de que el desarrollador, acostumbrado normalmente a tener una visión bastante reducida a aspectos \textbf{técnicos}, sea consciente de la \textbf{necesidad de implicar en el desarrollo otras voces}. Dichas voces quizás no posean los conocimientos técnicos necesarios para ''implementar'' la idea, pero sí que aportan una visión de conjunto más amplia que ayuda a entender mejor las necesidades reales del usuario, que a fin de cuentas es el destinatario de la aplicación. Realizar por tanto \textbf{entrevistas y posteriores evaluaciones} para tener en cuenta la opinión del futurible ''cliente'' nos ha enfrentado a una realidad necesaria para cualquier desarrollador que quiera conseguir un producto serio. Más aun cuando \textbf{nos planteamos DemoCritics como una aplicación que trascienda el alcance del Trabajo de Fin de Grado y se convierta en una herramienta útil en el futuro próximo.}

Los resultados obtenidos se plasman en el estado actual de DemoCritics. Por vez primera la ciudadanía tiene al alcance de sus manos un \textbf{espacio} donde consultar \textbf{todas las opciones políticas} (vengan de donde vengan), aprender sobre ellas, debatirlas o valorarlas. Así como también participar en un \textbf{proceso participativo} de propuestas ciudadanas, dándole la posibilidad de dar visibilidad a su propuesta o colaborar en el desarollo de otras. Se explota también el potencial social de la aplicación mediante la posibilidad de opinar, realizar comentarios y marcar como favorito en todo el contenido de la aplicación.

Se pone así tambien la \textbf{tecnología de colaboración en tiempo real} a disposición de la comunidad de desarrolladores de \textbf{Software libre} en Android para que hagan uso de estas funcionalidades en sus futuras aplicaciones. Así como también la aplicación que contiene la plataforma de Software libre de \textbf{participación ciudadana} de DemoCritics, que cualquier usuario podría \textbf{adaptar a las necesidades} y opciones de su región, país o comunidad.

Existe todavía \textbf{margen de mejora}, pues la presente plataforma es una primera versión de un proyecto en el que pretendemos \textbf{seguir investigando y desarrollando} para ponerla en práctica en un escenario real de cara a las \textbf{próximas elecciones generales} que se celebrarán durante los proximos meses.


