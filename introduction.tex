\newpage
\thispagestyle{sectioned}
\pagenumbering{arabic}
\chapter{Introducción}

A continuación se exponen de forma resumida los objetivos que se persiguen al realizar este Trabajo de Fin de Grado y un esquema de la estructura de esta Memoria.
 
\section{Objetivos del Proyecto}

El objetivo principal de este proyecto es desarrollar una aplicación android de utilidad social y que haga uso de una tecnología poco usada en estas plataformas móviles como es la colaboración en tiempo real, que desde hace unos años sí que viene estando más presente en plataformas web. Para ello primero se evaluará la tecnología existente en el proyecto SwellRT (basado en Apache Wave) y se adaptará dicha tecnología para poder hacer uso de ella de forma nativa desde Android. Después se discutirán posibles ideas de aplicación que puedan hacer uso de esta tecnología y se diseñará y desarrollará una primera versión estable de esa aplicación que pueda ser evaluable por usuarios. También se estudiará la viabilidad de seguir desarrollando a futuro la aplicación con vistas a conseguir un producto que podamos poner a disposición del público en el marketplace de Android.

\begin{itemize}
  \item {
    1ª parte: Migración de Wave (SwellRT) a Android
    \begin{itemize}
      \item Estudiar la implementación web actual de SwellRT en Java y GWT.
      \item Adaptar la implementación actual para hacer uso de las características nativas de Android.
      \item Probar con una aplicación sencilla que el resultado funciona.
    \end{itemize}
  }
  \item {
    2ª parte: Creación de la aplicación Android.
    \begin{itemize}
      \item Evaluar posibles ideas de aplicación y estudiar su viabilidad.
      \item Diseñar la aplicación mediante una metodología basada en el Diseño Guiado por Objetivos.
      \item Implementar la aplicación.
      \item Testear y evaluar con usuarios el resultado.
    \end{itemize}
  }
\end{itemize}

\section{Estructura del Documento}

En esta memoria hemos querido reflejar los aspectos más destacados del proceso de desarrollo de este Trabajo de Fin de Grado. Aunque a lo largo del documento se detallen aspectos técnicos queremos dejar patente que no se trata de un tutorial de cómo funciona Wave o Android. Aunque a veces se entre en mayor detalle por cuestiones de claridad a la hora de entender el funcionamiento de la tecnología, se anima al lector si quiere profundizar más en el tema a que haga uso de las múltiples referencias bibliográficas que se incluyen en este documento.  

A lo largo de este documento el lector se encontrará con una estructura basada en capítulos en los que se irán detallando distintas facetas del desarrollo del proyecto: 

\begin{itemize}
  \item \textbf{Capítulo 2 - Construyendo la idea de la aplicación DemoCritics:} se expone cómo y por qué se llegó a la idea de esta aplicación tras haber realizado la migración de SwellRT a Android.
  \item \textbf{Capítulo 3 - Estado del Arte:} se hace un pequeño análisis de las características de actuales soluciones software que hacen uso de tecnologías de colaboración en tiempo real y de aplicaciones dedicadas a la participación política.
  \item \textbf{Capítulo 4 - Tecnologías del Proyecto:} se repasan las tecnologías y herramientas utilizadas durante todo el desarrollo del proyecto.
  \item \textbf{Capítulo 5 - Metodologías del Proyecto:} se detallan el proceso de migración de SwellRT y las metodologías utilizadas el diseño, implementación y evaluación de DemoCritics.
  \item \textbf{Capítulo 6 - Arquitectura del Proyecto:} se indaga en aspectos técnicos destacados de la organización de DemoCritics.
  \item \textbf{Capítulo 7 - Resultados, Conclusiones y Trabajo Futuro:} se discuten los resultados obtenidos con el objeto de sacar algunas conclusiones. Se discuten los siguientes pasos a realizar con el objetivo  de mejorar o cambiar DemoCritics.
\end{itemize}
