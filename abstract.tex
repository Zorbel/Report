\newpage
\renewcommand{\thepage}{\Roman{page}}
\setcounter{page}{9}
\chapter*{Abstract}
\addcontentsline{toc}{chapter}{\numberline{}Abstract}
Abstract in English.
\vfill
{\large \bf Keywords:}\\
{\large Collaborative Development, Politics, Democracy, Citizen Participation, Real-Time, Android, Apache Wave, Elections, Election Programme, Proposals. }

\newpage
\renewcommand{\thepage}{\Roman{page}}
\setcounter{page}{10}
\chapter*{Resumen}
\addcontentsline{toc}{chapter}{\numberline{}Resumen}
Actualmente vivimos en una sociedad de transición democrática en la que las personas comienzan a adquirir un papel determinante en el terreno político. Son cada vez más los movimientos sociales que proliferan a raíz del interés generado por participar en politica. Esta participación se ha visto canalizada por la aparición de nuevas tecnologías que permiten establecer nuevas formas para que las personas se organicen y expresen su opinión. La gran mayoría de estas nuevas herramientas se desarrollan en un ámbito cercano a plataformas interconectadas en red y basadas en una interfaz web. Encontrándonos en un momento sujeto a varias citas electorales que propicia el aumento del interés de las personas por involucrarse en política, el desarrollo de este tipo de herramientas se hace cada vez mas necesario.

Teniendo en cuenta este contexto, se plantea el desarrollo de una aplicación que proporcione acceso a nuevas formas de participación desde dispositivos móviles. Dicha aplicación tendrá dos objetivos claros: poner en un primer plano los programas electorales (que normalmente pocas personas leen) incentivando su lectura y debate; y abrir un espacio común en el que la ciudadanía exprese sus propuestas alternativas a las soluciones expuestas por los partidos tradicionales que les representan.

Nos basaremos en el uso de tecnologías open-source ya existentes de edición colaborativa en tiempo real, adaptándolas a dispositivos móviles. Concretamente se explorará el uso de la plataforma web Apache Wave, llevando a cabo un proceso de migración que permita explotar su potencial en dispositivos basados en Android. Haciendo uso de esta tecnología los usuarios tendrán la oportunidad de desarrollar textos colaborativos en tiempo real.

Como resultado de este trabajo se desarrollará una primera versión de la aplicación que, a modo de prueba de concepto, haga uso de las funcionalidades antes descritas. De esta manera el objetivo final que nos planteamos es poner en práctica su uso durante futuras citas electorales.

\vfill
{\bf Palabras Clave:}\\
{Desarrollo Colaborativo, Política, Democracia, Participación Ciudadana, Tiempo Real, Android, Apache Wave, Elecciones, Programas Electorales, Propuestas.}

\newpage
\thispagestyle{empty}
\mbox{}

