\newpage
\thispagestyle{sectioned}
\chapter{Metodología del Proyecto}

A continuación se exponen las diferentes metodologías que hemos utilizado durante el desarrollo de este \textit{Trabajo de Fin de Grado}.

\section{Uso de Software Libre}

Para el desarrollo de todo el software que compone el proyecto siempre se ha utilizado una aproximación de software libre. Esto permite contribuir al común y que cualquier persona pueda hacer uso de nuestro código si lo necesita. Se ha usado software libre tanto con las herramientas necesarias para el desarrollo de código y documentación como para el resultado final de la aplicación. Con esto se da la posibilidad a que otros usuarios puedan visualizar el código desarrollado o utilizarlo libremente. Para ello se realizaron aportaciones a toda la comunidad subiendo el código del proyecto a \textbf{GitHub} bajo una licencia \textbf{GNU GPLv3} \cite{ref:GPLv3}.

En la tabla \ref{fig:tableLicenses} se expone un breve resumen del software libre utilizado y las licencias que poseen.

\begin{table}[h]
\centering
\begin{tabular}{|c|c|}
\hline
{\bf Software} & {\bf Licencia}                 \\ \hline
Eclipse        & Eclipse Public License         \\ \hline
Android Studio & Apache License 2.0             \\ \hline
Laravel        & MIT License                    \\ \hline
Apache Wave    & Apache License                 \\ \hline
phpMyAdmin     & GNU GPLv2                      \\ \hline
MySQL          & GNU GPL                        \\ \hline
PHP            & PHP License                    \\ \hline
Android        & Apache License 2.0 y GNU GPLv2 \\ \hline
Java           & GNU GPL                        \\ \hline
OpenShift      & Apache License 2.0             \\ \hline
\end{tabular}
\caption{Software libre utilizado durante el desarrollo.}
\label{fig:tableLicenses}
\end{table}

\section{Patrones de Diseño}

La figura \ref{fig:tablePatrones} representa los patrones utilizados en el desarrollo de todo el proyecto.

\begin{table}[!]
\centering
\caption{Patrones de diseño empleados en el proyecto.}
\label{my-label}
\begin{tabular}{|c|m{5cm}|l|c|}
\hline
{\bf Patrón} & \multicolumn{1}{c|}{{\bf Decripción}}                                                                                                & \multicolumn{1}{c|}{{\bf Utilización}} & {\bf Referencia} \\ \hline
MVC          & Modelo-Vista-Controlador. Separa la lógica del modelo y la vista, que se comunican a través del controlador.                         & Servicio Web Laravel.                  & Sección          \\ \hline
Singleton    & Mantiene una única instancia de un objeto, pudiento instanciarse desde cualquier clase.                                              & Aplicación Android.                    & Sección          \\ \hline
Factoría     & Permite crear, mediante una interfaz, conjuntos o familias de que dependen, mutuamuente, sin especificar cual es el objeto concreto. & Servicio SellRT.                       & Sección          \\ \hline
\end{tabular}
\label{fig:tablePatrones}
\end{table}

\section{Migración de Wave/SwellRT a Android}

PROCESO DE MIGRACION.

\section{Diseño de la Aplicación}

  \subsection{\textit{Brainstorming}}
  
  El \textit{brainstorming} \cite{ref:bookBrainStorming} (o lluvia de ideas en español) es una técnica de trabajo en grupo que persigue que todos sus integrantes se junten para generar ideas originales en un ambiente distendido. El objetivo es primar la cantidad por encima de la calidad de éstas para después de la sesión realizar un proceso de selección de las más interesantes. De esta manera pueden salir ideas que a priori podrian resultar absurdas pero que otros integrantes del grupo pueden aprovechar para que surjan otras nuevas.  
  
  \subsection{Diseño Guiado por Objetivos (DGO)}\label{ssec:dgoDesign}
  
  Para elaborar el diseño de la aplicación, nos hemos basado en la metodología del Diseño Guiado por Objetivos (\textbf{DGO} o \textit{Goal-Directed Design}), que implementa el proceso de la Ingeniería de la Usabilidad propuesto por Alan Cooper \cite{ref:bookAlanCooper}. Este proceso constará de las siguientes fases:

\begin{enumerate}

\item \textbf{Investigación}

Esta fase consistirá en la realización de estudios para obtener datos cualitativos sobre los usuarios y/o reales de la aplicación y cuáles son sus necesidades. Se realizarán tareas para comprender a los usuarios, saber sus inquietudes y lograr empatía. A lo largo de esta fase se irán identificando patronos de comportamiento que sugerirán los objetivos y motivaciones del usuario. Por último se realizarán estudios de mercado, revisiones y auditorías que ayudarán al diseñador a comprender el dominio, el modelo y las restricciones técnicas que el sistema debe cumplir.

\item \textbf{Modelado}

A lo largo de esta fase se utilizarán los datos provenientes de la fase previa para crar los modelos del dominio y los usuarios. En esta parte se crearán las \textit{personas}, aquetipos de usuarios que contienen información sobre obejtivlos, motivaciones y comportamientos de los usuarios con el sistema. El resultado final de esta fase serán los tipos de \textit{persona} que representarán a los usuarios del sistema, y que más adelante serán utilizados fases para proporcionar algún tipo de \textit{feedback}.

\item \textbf{Definición de Requisitos}

Durante esta fase se utilizarán las personas y los datos de la fase anterior, para crear los \textit{escenarios} de contexto e identificar los requisitos o necesidades del usuario. Estos requisitos serán definidos en tres componentes: objeto, acciones y contexto. También se definirán requisitos relacionados con el negiocio, la apclicación, requisitos técnicos, etcétera.

\item \textbf{Definición del framework de Diseño}

En esta fase se creará el concepto general del sistema, definiendo los comportamientos y diseño visual. Identificaremos el \textbf{framework de interacción}, como un concepto de diseño estable que define la estructura del sistema a partir de patrones y principios de diseño. Por último, se definirá el \textit{framework visual}, como el aspecto visual de la aplicacion (diseño, tipografía, colores, iconos, etcétera).

\end{enumerate}

Queremos insistir eso sí en que únicamente nos hemos basado en esta metodología para seguir el proceso del diseño de la aplicación. No seguiremos todas las fases de esta metodología al pie de la letra ya que el alcance en tiempo de este proyecto escapa a una metodología tan formal y laboriosa (que implica gran cantidad de pruebas y procesos) como esta.

\section{Implementación de la Aplicación}

En esta fase se llevó a cabo la implementación de un prototipo final de la aplicación basado en el estudio de los resultados de la fase de diseño previa. Además se diseñó e implementó también el servidor que alojaría la base de datos y se comunicaría con la aplicación. Durante el desarrollo del código del sistema se han utilizado las metodologías que se describen a continuación.

\subsection{Control de versiones}

Para mantener una copia de las versiones locales, se ha utilizado utilizado GIT como herramienta de control de versiones. Realizando \textit{commits} y subiendo los cambios de cada versión en un repositorio público alojado en GitHub.

\subsection{GitHub}

Para compartir el código del sistema y visualizar los cambios de cada commit, se ha utilizado una organización pública donde se almacenan todos los repositorios de sus distintas partes. La organización está disponible en el siguiente enlace: \url{https://github.com/Zorbel}.

\subsection{Reparto de tareas}

Durante la implementación del proyecto se dividieron las tareas en dos grupos principales. Por un lado se tenía la parte de \textit{back-end}, formada por la API Rest desarollada en Laravel y la base de datos. Mientras que por otro lado se encontraba la parte de \textit{front-end} formada por la aplicación en Android. De cada una de estas partes y debido a la experiencia previa con las tecnologías involucradas se encargó uno de los componentes del grupo. De esta manera Jaime trabajó más en el desarrollo del back-end y Javier en el del front-end. No obstante ambos revisaban el código del otro para verificar su correcto funcionamiento. El desarrollo inicial en Wave/SwellRT y su implementación final en la aplicación se produjo en paralelo por ambos componentes del grupo.

\subsection{Revisiones de código}

Cada vez que se completaba una funcionalidad se subía a \textit{GitHub} y se realizaba una pequeña revisión de código para verificar cómo se había implementado el objetivo a desarrollar. De esta forma se evitaba malinterpretar algunos aspectos que durante la implementación pueden diferir de su planteamiento inicial. También se recibieron revisiones de código externas por parte de los directores del Trabajo de Fin de Grado, que aportaron una visión más eficiente y organizada del código desarrollado.

\subsection{Evaluaciones de usabilidad}

Se recibió una evaluación de usabilidad externa por parte del director del TFG Pablo, que valoraba aspectos a mejorar de la Intefaz de Usuario (UI). Esta evaluación sivió para modificar algunos aspectos de representación gráfica que mejorararían la interpretación de los elementos por parte del usuario.

