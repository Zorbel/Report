\newpage
\thispagestyle{sectioned}
\chapter{Metodología del Proyecto}

A continuación se exponen las diferentes metodologías que hemos utilizado durante el desarrollo de este \textit{Trabajo de Fin de Grado}.

\section{Uso de Software Libre}

Para el desarrollo de todo el software que compone el proyecto siempre se ha utilizado una aproximación de software libre. Esto permite contribuir al común y que cualquier persona pueda hacer uso de nuestro código si lo necesita. Se ha usado software libre tanto con las herramientas necesarias para el desarrollo de código y documentación como para el resultado final de la aplicación. Con esto se da la posibilidad a que otros usuarios puedan visualizar el código desarrollado o utilizarlo libremente. Para ello se realizaron aportaciones a toda la comunidad subiendo el código del proyecto a \textbf{GitHub} bajo una licencia \textbf{GNU GPLv3} \cite{ref:GPLv3}.

A continuación se expone un breve resumen del software libre utilizado y las licencias que poseen:

\begin{table}[h]
\centering
\begin{tabular}{|c|c|}
\hline
{\bf Software} & {\bf Licencia}                 \\ \hline
Eclipse        & Eclipse Public License         \\ \hline
Android Studio & Apache License 2.0             \\ \hline
Laravel        & MIT License                    \\ \hline
Apache Wave    & Apache License                 \\ \hline
phpMyAdmin     & GNU GPLv2                      \\ \hline
MySQL          & GNU GPL                        \\ \hline
PHP            & PHP License                    \\ \hline
Android        & Apache License 2.0 y GNU GPLv2 \\ \hline
Java           & GNU GPL                        \\ \hline
OpenShift      & Apache License 2.0             \\ \hline
\end{tabular}
\caption{Software libre utilizado durante el desarrollo.}
\end{table}

\section{Patrones de Diseño}

TABLA

\section{Migración de Wave/SwellRT a Android}

PROCESO DE MIGRACION.

\section{Diseño de la Aplicación}

BRAINSTORMING Y DGO

\section{Implementación de la Aplicación}

En esta fase se llevó a cabo la implementación de un prototipo final de la aplicación basado en el estudio de los resultados de la fase de diseño previa. Además se diseñó e implementó también el servidor que alojaría la base de datos y se comunicaría con la aplicación. Durante el desarrollo del código del sistema se han utilizado las metodologías que se describen a continuación.

\subsection{Control de versiones}

Para mantener una copia de las versiones locales, se ha utilizado utilizado GIT como herramienta de control de versiones. Realizando \textit{commits} y subiendo los cambios de cada versión en un repositorio público alojado en GitHub.

\subsection{GitHub}

Para compartir el código del sistema y visualizar los cambios de cada commit, se ha utilizado una organización pública donde se almacenan todos los repositorios de sus distintas partes. La organización está disponible en el siguiente enlace: \url{https://github.com/Zorbel}.

\subsection{Reparto de tareas}

Durante la implementación del proyecto se dividieron las tareas en dos grupos principales. Por un lado se tenía la parte de \textit{back-end}, formada por la API Rest desarollada en Laravel y la base de datos. Mientras que por otro lado se encontraba la parte de \textit{front-end} formada por la aplicación en Android. De cada una de estas partes y debido a la experiencia previa con las tecnologías involucradas se encargó uno de los componentes del grupo. De esta manera Jaime trabajó más en el desarrollo del back-end y Javier en el del front-end. No obstante ambos revisaban el código del otro para verificar su correcto funcionamiento. El desarrollo inicial en Wave/SwellRT y su implementación final en la aplicación se produjo en paralelo por ambos componentes del grupo.

\subsection{Revisiones de código}

Cada vez que se completaba una funcionalidad se subía a \textit{GitHub} y se realizaba una pequeña revisión de código para verificar cómo se había implementado el objetivo a desarrollar. De esta forma se evitaba malinterpretar algunos aspectos que durante la implementación pueden diferir de su planteamiento inicial. También se recibieron revisiones de código externas por parte de los directores del Trabajo de Fin de Grado, que aportaron una visión más eficiente y organizada del código desarrollado.

\subsection{Evaluaciones de usabilidad}

Se recibió una evaluación de usabilidad externa por parte del director del TFG Pablo, que valoraba aspectos a mejorar de la Intefaz de Usuario (UI). Esta evaluación sivió para modificar algunos aspectos de representación gráfica que mejorararían la interpretación de los elementos por parte del usuario.

