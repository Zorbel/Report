\newpage
\thispagestyle{sectioned}
\chapter{Construyendo la idea de la aplicación: DemoCritics}

\section{Introducción}

Tras haber adaptado (Ver Sección \ref{sec:migration}) la tecnología de Wave/SwellRT a Android que soportaría el núcleo de nuestra aplicación, era hora de decidir qué aplicación le íbamos a dar de cara a desarrollar una aplicación Android que hiciera uso de ello. Era importante tener en cuenta las características que nos ofrecía Wave:

 - Edición colaborativa.

 - Consistencia en Tiempo Real.
 
\subsubsection{Brainstorming de ideas}

Un \textit{brainstorming} \cite{ref:bookBrainStorming} (o lluvia de ideas en español) es una técnica de trabajo en grupo que persigue que todos sus integrantes se junten para generar ideas originales en un ambiente distendido. El objetivo es primar la cantidad por encima de la calidad de éstas para después de la sesión realizar un proceso de selección de las más interesantes. De esta manera pueden salir ideas que a priori podrian resultar absurdas pero que otros integrantes del grupo pueden aprovechar para que surjan otras nuevas.  

Después de considerar posibles ideas de implementaciones que podrían hacer uso de estas características, decidimos realizar una sesión de \textit{brainstorming} junto a nuestros directores de proyecto para identificar ideas potenciales. En esta sesión aparecieron temas tan variados como wikis colaborativas, aplicaciones de inteligencia artificial, aportaciones colaborativas en política, edición de vídeos y música, cursos de formación colaborativos, visualizacion de mapas, etc.

	\begin{figure}[H]
        \centering
        \begin{subfigure}[b]{0.7\textwidth}
                \includegraphics[width=\textwidth]{Media/Captures/brainstorming.jpg}
                \caption{Idea de Aplicación}
                \label{fig:brainstormingApp}
        \end{subfigure}
        ~
        \begin{subfigure}[b]{0.7\textwidth}
                \includegraphics[width=\textwidth]{Media/Captures/appname.jpg}
                \caption{Posibles Nombres}
                \label{fig:brainstormingName}
        \end{subfigure}
        \caption{Capturas del brainstorming}\label{fig:brainstormingCaptures}
	\end{figure}
	


Con un gran repertorio de ideas expuestas en la sesión, decidimos centrarnos primero en descartar aquellas que a nosotros no nos motivaba llevar a cabo. De esta manera nos quedamos con cuatro ideas fundamentales a desarrollar en nuestra aplicación: Política, Música, Inteligencia Artificial y Mapas. Centrándonos ahora solo en estos temas, surgieron varias ideas colaborativas como: desarrollar documentos políticos, programas electorales, comunicación entre colectivos en tiempo real, aprendizaje de música, edición de partituras y obras, aplicaciones colaborativas con inteligencia artificial, edición de mapas en tiempo real, lexicalización, etc.

Finalmente, ya que ambos teníamos interes por la política, decidimos realizar una aplicación colaborativa relacionada con dicho mundo. En esta aplicación podríamos recurrir a la edición de contenidos en tiempo real, ya fueran propuestas políticas, programas electorales u otro tipo de documentos. Más adelante también podríamos hacer uso incluso de alguna herramienta de Inteligencia Artificial para automatizar algunas tareas o realizar recomendaciones sociales.

Lo que sí que tuvimos claro desde el principio es la idoneidad del momento actual para desarrollar una aplicación de temática política, dado que nos encontramos en año electoral. Nos propusimos el objetivo de desarrollar algo que pudiera tener cierta repercusión y utilidad en las próximas citas electorales de este año 2015. Intentariamos pensar en la aplicación no solo como un Trabajo de Fin de Grado sino como algo que pudiéramos llevar más allá y que resultara útil a la sociedad.

Por otro lado, queriamos elegir un nombre para la aplicación que fuera a la vez atractivo y significativo de la participación ciudadana que representa. Para esto hicimos tambien una sesión de \textit{brainstorming} con varias personas, de la cual sacamos varias posibilidades por el nombre. Tras someter a votación estos nombres nos quedamos con el que más gustó a todos: \textbf{DemoCritics}.




\section{Adentrándonos en la idea}

La idea a desarrollar está generada en una época en la que la política parece haber despertado el interés de una parte considerable de la ciudadanía. Podría ser por tanto una herramienta útil para participar en temas políticos de forma sencilla y atractiva. Dejando así atrás los tópicos a menudo escuchados de \textit{"yo no entiendo de política"}, \textit{"la política es aburrida"}, \textit{"no sé a quién votar"} o \textit{"no he leído nunca un programa electoral"} entre otros.

La herramienta ofrecería una nueva forma de participar en la política y de llevar a los ciudadanos los programas electorales expuestos por las diferentes formaciones políticas. De forma que, para potenciar el uso social de la aplicación, los ciudadanos podrían leer aquellos puntos de los programas más vistos, debatidos, comentados, etc. Así, cualquier usuario tendría a su disposición todos los programas electorales en su bolsillo, por lo que no tendría que ir a la página web de cada formación política y descargarse un documento de 200 páginas. Pensamos que esta forma tradicional de presentar un programa político en un solo documento en un mundo donde las posibilidades de comunicarnos se han desarrollado exponencialmente mediante las nuevas tecnologías no es la mejor manera de generar interés por su lectura y la implicación en política de las personas.

Por otra parte, y teniendo en cuenta la tendencia actual de los nuevos movimientos ciudadanos de elaborar programas políticos en base a propuestas de los ciudadanos, \textbf{la aplicación también debía ofrecer alguna manera de realizar Propuestas y debatirlas entre todos}. De esta forma tanto la ciudadanía como las formaciones políticas podrían saber en cualquier momento cuáles son las principales preocupaciones de los ciudadanos y qué medidas o soluciones proponen para resolverlas. \textbf{Además pensamos que podríamos aprovechar las características de Wave para realizar estas Propuestas de forma colaborativa y en tiempo real}, aportando un valor diferenciador respecto a las actuales soluciones desarrolladas para web (Ver sección \ref{ssec:artProposals}).

Por tanto, desde un primer punto de vista subjetivo, la aplicación quedó dividida en dos partes. Por un lado tendríamos la presentación estructurada de los programas políticos que presentan las formaciones políticas. Y por otro todas las propuestas que elaboran de forma colaborativa los ciudadanos, ya sea individualmente o en colectivos sociales.



