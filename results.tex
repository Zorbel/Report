\newpage
\thispagestyle{sectioned}
\chapter{Resultados y Trabajo Futuro}

En este capítulo se analizarán y discutirán los resultados en conjunto de todo el Trabajo de Fin de Grado, así como algunas líneas de trabajo futuro para este proyecto.

Todo el código de Software libre desarrollado por nosotros está disponible en el GitHub de la organización del proyecto:

\url{https://github.com/Zorbel}

\section{Discusión de Resultados}

SE ha desarrollado \textbf{DemoCritics}, una plataforma para dispositivos móviles Android cuyo objetivo es \textbf{juntar en un único sitio el mensaje de la política tradicional} (visto en forma de su Programa Político) \textbf{con el mensaje de los ciudadanos (visto en forma de Propuestas)}.

No podemos olvidar del ''leitmotiv'' inicial de este proyecto, utilizar las capacidades de colaboración en tiempo real de Wave/SwellRT en Android. Así, nacieron las \textbf{propuestas colaborativas} (o ''abiertas''). Propuestas en las que un usuario al crearlas dejaba alguno de los dos campos (''¿Cómo la llevaría a cabo?'' y ''¿Cómo la financiaria?'') en blanco para permitir a otros usuarios que le echaran una mano mediante la edición colaborativa y en tiempo real de esa parte de la propuesta. De esta manera utilizamos Wave/SwellRT en un contexto más lógico que el de redactar la propuesta entera, pues al menos dichos campos están sujetos a menos subjetividad que la declaración de intenciones de una propuesta más al uso de las que se encuentran en las plataformas actuales. Además, solamente el usuario inicial de la propuesta es el único capaz de convertir la propuesta en ''definitiva'', cerrando la posibilidad de su edicion con Wave cuando estime que la comunidad de ''expertos'' ha podido ayudarle a proponer algo coherente y viable. 

Por otro lado, se ha querido aprovechar el auge de las redes sociales para ''socializar'' la plataforma. De esta forma los usuarios pueden generar contenido e interés en el uso de la aplicación. Pueden dar su opinión mediante el uso de ''indicadores sociales'' similares a los ''Me gusta'' o ''No me gusta'' tan presentes en las redes sociales. También pueden realizar sus propios comentarios para generar debate y marcar contenido como favorito. Todo ello aplicado tanto a secciones de programas políticos como a propuestas hechas por otros usuarios. Esto, por otro lado, nos permite en cierta manera ''cuantificar'' el interés de las personas, lo cual siempre resulta útil para clasificar el contenido en forma de ''tops'' de lo más valorado, lo más comentado, etc. 

Aunque no solo es de utilidad para elaborar dichos filtros de visualización que atraigan la atención de los usuarios, sino que también los propios partidos políticos podrían beneficiarse de ello para identificar sus propuestas políticas más controvertidas o más valoradas. Al final se trata de una especie de ''red social'' aplicada al ámbito político en el que cualquiera de los actores puede intervenir y beneficiarse del contenido que genera la comunidad. 

También se utiliza un sistema de categorización tanto de secciones como de propuestas para llevar a cabo esa unión entre los programas politicos y las propuestas que tan necesaria nos parecía para fomentar tanto la lectura de secciones como la elaboración de propuestas. De hecho, esto atraería a usuarios que, lejos de querer leerse un programa político, lo que buscan es conocer el contenido de una determinada categoría de su interés. Aunque actualmente existen 6 categorías (Sanidad, Educación, Empleo, Vivienda, Impuestos y Cultura) la intención es que existan más y que incluso los propios usuarios interesados puedan crear sus propias categorías, tal y como se discutirá después en el trabajo a futuro. 

Desde un punto de vista técnico el desarrollo de DemoCritics aporta una plataforma que es independiente de la tecnología del cliente, pues se ha desarrollado un Servicio Web REST que responde a peticiones HTTP accesibles desde cualquier tipo de cliente que pueda realizar este tipo de peticiones. Además se trata de un proyecto en el que se aporta una solución de Software libre que no existía previamente, pues las opciones de colaboración en tiempo real en Android pasaban únicamente por soluciones privativas como el Real Time API de Google. Su código está disponible en GitHub y cualquiera puede estudiarlo, contribuir, copiarlo y utilizarlo
libremente. 

\subsection{Resultados de la Evaluación con usuarios} \label{ssec:evaluationResults}

En líneas generales se puede decir que la aplicación ha sido recibida de forma positiva por la mayor parte de los usuarios, siendo la innovación en torno a la idea el factor potencial de la aplicación. Algunos de los usuarios estaban familiarizados con herramientas similares en otros entornos (Web), y valoraron positivamente su traslado a una plataforma móvil.

Los usuarios apreciaron la posibilidad de poder leer los programas en el móvil, como si de un programa de bolsillo se tratase, y poder valorarlos y opinar sobre ellos. No obstante, la apariencia o representación de los programas es algo que no ha atraído demasiado su atención. Por ello, se deberá trabajar más en otra posible representación gráfica que capte más la atención del usuario.

La interacción entre las tareas y la transición de las pantallas, no ha supuesto un problema para los usuarios. A excepción de las propuestas colaborativas, pues habría que rediseñar el concepto para hacer la edición en tiempo real más \textit{amigable} en consonancia con otras herramientas similares.

Para la siguiente interacción del proceso de desarrollo, se proponen los siguientes puntos por orden de prioridad, teniendo en cuenta las recomendaciones y evaluaciones de los usuarios:

\begin{enumerate}
 \item Cambiar el aspecto de los iconos de la pantalla inicial por otros más representativos a la actividad que respresentan.
 \item Dar más visibilidad a los botones de opinión en secciones y propuestas.
 \item Rediseñar la visualización de las secciones para una lectura más cómoda y diferenciar bien los elementos con los que puede interactuar el usuario de los que no.
 \item Estudiar la visibilidad del índice de programa expansible en cada sección.
 \item Mejorar el concepto de propuestas colaborativas con diferenciación de usuarios por colores o etiquetas para mejorar su comprensión durante su edición.
 \item Estudiar la creación de nuevas categorías que representen mejor las prouestas y secciones de la aplicación.
 \item Ofrecer más funcionalidades de personalización al usuario.
\end{enumerate} 
 
\section{Reparto de Trabajo entre los componentes del grupo} 

DemoCritics es un proyecto desarrollado por dos personas. Para su desarrollo hemos elegido una metodología de control de versiones que nos permite repartir las tareas de forma eficaz (cualquiera de los dos puede ver en pocos pasos lo hecho por el otro) y eficiente (lleva poco tiempo subir o descargarse los cambios). Para ello creamos una organización en GitHub llamada ''Zorbel'' y gestionamos con git los cambios en el código de cualquiera de las partes que conforman DemoCritics. Además, esto nos permitió también facilitar la revisión del código por parte de los profesores del proyecto. 

En general, todas las tareas se han llevado a cabo en forma paralela y con una carga de trabajo equitativa entre ambos miembros. A continuación se detallan un poco más estas tareas:

- \textbf{Migración de SwellRT a Android}: Tanto la investigación del funcionamiento de la tecnología como la metodología de migración se llevaron a cabo por ambos miembros. Estos cambios se contrastaron en reuniones con el director Pablo Ojanguren (desarrollador de SwellRT) a medida que se producían. De poner el API de trabajo con waves de SwellRT sobre el servicio migrado se encargó Pablo.

- \textbf{Desarrollo de la idea de la app}: los dos miembros del proyecto nos reunimos con los directores Samer y Pablo para hacer una sesión de brainstorming y dar con una idea de aplicación que a ambos nos gustaría realizar y que hiciera uso de SwellRT.

- \textbf{Diseño e Investigación}: ambos hicimos reuniones periódicas para discutir ideas iniciales, planificar la investigación y diseñar prototipos. Durante las entrevistas de investigación procuramos estar ambos presentes para exponer las idea, discutirlas y enseñarles los prototipos básicos a los entrevistados. Posteriormente nos reunimos para sacar conclusiones y modificar los prototipos. En definitiva, toda la fase basada en el Diseño Guiado por Objetivos la realizamos en conjunto.

- \textbf{Implementación de la aplicación}: En este caso sí que hubo un reparto de tareas más marcado por la poca experiencia previa que teníamos con distintas tecnologías. El diseño general de la aplicación, el Service REST y la Base de Datos lo discutimos entre ambos. A la hora de implementarlo uno de nosotros (Javier) conocía mejor la arquitectura de Android y se dedicó a elaborar el funcionamiento del cliente mientras que el otro (Jaime) había trabajado previamente con Laravel y se dedicó a investigar y desarrollar el Service REST y la conexión con la Base de Datos. No obstante, ambos pusimos en común los cambios realizados, de manera que los dos tuvimos un conocimiento de cómo funcionaba la aplicación en su conjunto. Durante la última fase del proyecto de prefeccionamiento de detalles de la aplicación ambos trabajamos para mejorar el cliente Android.

- \textbf{Evaluación con Usuarios}: planificamos juntos la dinámica de las evaluaciones y realizamos dos evaluaciones con usuarios cada uno de un total de 4 evaluaciones, procurando poner luego en común los resultados para discutirlos y sacar conclusiones.

- \textbf{Elaboración de la memoria}: para la redacción de esta memoria procedimos a realizar varios posibles índices para su estructuración hasta que obtuvimos un primer esqueleto de su estructura. Posteriormente nos repartimos su redacción, procurando verificar con el compañero el contenido de todo lo escrito. Esto se vió facilitado por el uso también para la memoris de git  GitHub como sistema de control de versiones (ver organización del proyecto en GitHub). En general, habremos escrito ambos aproximadamente la mitad del contenido de este documento cada uno (siendo siempre verificado por el otro).

La tabla \ref{fig:tableTasks} muestra el reparto colaborativo de tareas entre los integrantes del grupo.

\begin{table}[!]
\centering
\caption{Reparto colaborativo de tareas.}
\label{fig:tableTasks}
\begin{sideways}
\begin{tabular}{|m{3cm}|m{5.5cm}|m{5cm}|m{5cm}|}
\hline
\multicolumn{1}{|c|}{{\bf Tarea}} & \multicolumn{1}{c|}{{\bf Descripción}}                                                                                                   & \multicolumn{1}{c|}{{\bf Jaime}}                                                                                                   & \multicolumn{1}{c|}{{\bf Javier}}                                                                            \\ \hline
Migración de SwellRT Android      & Migrar la plataforma en desarrollo SwellRT a Android, adaptando las clases y comoponentes principales para su funcionamiento en Android. & Adaptar librería HTTP para relalizar peticiones al servidor y establecer la conexión.                                              & Establecer la conexión bidireccional a través de WebSockets para el intercambio de contenido en tiempo real. \\ \hline
Desarrollo de la idea aplicación  & Estudiar y valorar las posibles soluciones para el desarrollo de una aplicación móvil colaborativa en tiempo real.                       & Evaluar el desarrollo de aplicaciones colaborativas que utilizaran técnicas de Inteligencia Artificial, destinadas a la educación. & Promover el desarrollo de aplicaciones relacionadas con geolocalización en mapas y producción audivisual.    \\ \hline
Diseño e investigación            & Concretar entrevistas con personas de interés para el diseño de la aplicación, desarrollar los bocetos y definir su interacción.         & Seguimiento y análisis de las entrevisas y desarrollo y diseño de los bocetos a papel.                                             & Seguimiento y anáslisis de las entrevistas y definición de la interacción de los prototipos.                 \\ \hline
Implementación de la aplicación   & Desarrollo del cliente móvil en Android, diseño de la base de datos y desarrollo del Servicio Web.                                       & Diseño de la base de datos, desarrollo del Servicio Web y apoyo en el cliente Android.                                             & Diseño de la base de datos, desarrollo del cliente Android y apoyo en el Servicio Web.                       \\ \hline
Evaluación con usuarios           & Evaluaciones de la aplicación final a diferentes usuarios.                                                                               & Evaluaciones a personas con el perfil de activista social.                                                                         & Evaluaciones a personas con el perfil de ciudadano de a pie.                                                 \\ \hline
Elaboración de memoria            & Desarrollo de la memoria final del proyecto en \LaTeX\.                                                                                & Diseño, estructuración y redacción de la memoria.                                                                                  & Diseño, estructuración y redacción de la memoria.                                                            \\ \hline
\end{tabular}
\end{sideways}
\end{table}

Para el seguimiento del proyecto procuramos realizar mínimo una reunión al mes, ya fuera con Samer, con Pablo o con ambos. Cabe destacar que, si bien a veces no pudimos reunirnos en persona, el uso de herramientas de mensajería instantánea (con especial mención a Telegram) y de videollamada como Skype y Hangouts fue constante entre los involucrados en el proyecto.

\section{Trabajo Futuro}

Como ya se ha comentado con anterioridad la presente es una primera versión de DemoCritics que nos sirve para tener una primera prueba de concepto de algunas de sus funcionalidades y que haga uso de la migración de SwellRT a Android. En esta versión hemos podido evaluar el funcionamiento general de la idea y ponerla a prueba con algunos usuarios para mejorar su utilidad de cara a continuar su desarrollo. Nuestro objetivo en este momento es aprovechar la oportunidad que se nos presenta en este año electoral para poner realmente a prueba esta plataforma en las próximas elecciones generales. De esta forma también podriamos analizar su uso antes, durante y después de unas elecciones.

Para ello nos proponemos seguir poniéndonos en contacto con actores relacionados con el mundo de la política (activistas, políticos, periodistas, académicos, etc.) para realizar más entrevistas y someter la plataforma a nuevas iteraciones del proceso de diseño, implementación y evaluación con los que ir poco a poco mejorando sus características. De paso también iremos generando interés entre los entrevistados y publicitando su uso.

En cualquier caso, el proyecto está disponible en forma de software libre bajo licencia GNU GPLv3 que cualquiera puede estudiar, copiar y modificar para desarrollar sus propias ideas inspirado por las nuestras. 

\subsection{Mejoras a la versión actual}

Teniendo en cuenta los resultados de la evaluación con usuarios  (Ver sección \ref{ssec:evaluationResults} creemos que las líneas de trabajo futuro deberían ir orientadas a mejorar la experiencia de usuario, modificando aspectos de la interfaz gráfica que pudieron resultar más incomprensibles a las personas evaluadas, como los iconos de la aplicación y su significado. También habría que mejorar la visualización de las distintas secciones del programa, que resultó confusa para algunos de ellos. 

Desde el punto de vista del desarrollo colaborativo en tiempo real de propuestas, se debería estudiar la posibilidad de poder distinguir a los usuarios participantes en la Wave y lo que escriben mediante el uso de indicadores de color o etiquetas al estilo de otras plataformas web como Google Docs. 

\subsection{Nuevas características}

Durante el desarrollo del proyecto han ido surgiendo diversas ideas interesantes que se podrían aplicar en un futuro próximo dentro de la plataforma de DemoCritics. Algunas de las posibilidades son:

- Desarrollo de Encuestas: de intención de voto, de afinidad, etc. con posibilidad de obtener los resultados en tiempo real mediante SwellRT.

- Abrir la creación de categorías nuevas a los usuarios, de manera que demos más flexibilidad a la hora de crear y clasificar contenidos.

- Permitir que los usuarios hagan comparativas por temas entre las secciones de los distintos programas de partidos políticos. Buscar para ello la participación de colectivos sociales interesados.  

- Creación de una hemeroteca de programas políticos que permita navegar por programas políticos de anteriores elecciones. Útil por ejemplo para comprobar la evolución de las propuestas de los partidos.

- Integración con redes sociales y mensajería instantánea para compartir contenidos de la aplicación.

- Oraganizar jerárquicamente los usuarios, pudiendo existir grupos de usuarios que comparten las mismas inquietudes e intereses.

- Uso de inteligencia artificial para integrar sistemas recomendadores de contenido por afinidad o de puesta en contacto entre usuarios por intereses similares.
 
Asimismo, y gracias a la flexibilidad que nos permite el uso de un Service REST, se estudiará la posibilidad de \textbf{desarrollar un cliente web} para la aplicación, especialmente pensado para usuarios que están más familiarizados con la elaboración de textos largos, como los de las propuestas, usando un teclado. De esta manera enfocaríamos la aplicación móvil a la lectura de contenidos y generación de opiniones y comentarios, mientras que las propuestas se podrían redactar en el ordenador. Entendemos que la postura de uso de un móvil es para usarlo durante un espacio corto de tiempo, por lo que la existencia de un cliente web para ordenadores haría la generación de contenidos más serios y extensos, como las propuestas, mucho más cómoda.

Por otro lado, en el desarrollo de este proyecto hemos centrado la plataforma en los programas políticos, pero se podría utilizar para mostrar otros tipos de documentos estructurados susceptibles de ser sometidos a debate y opinión. Como por ejemplo normativas, leyes, manuales, etc.






