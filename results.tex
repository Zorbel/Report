\newpage
\thispagestyle{sectioned}
\chapter{Resultados, Conclusiones y Trabajo Futuro}

En este capitulo analizaremos y discutiremos los resultados en conjunto de todo el Trbajo de Fin de Grado, las conclusiones que se pueden sacar tras su desarrollo y las lineas de trabajo futuro para este proyecto.

PROPUESTAS, PARTICIPACION CIUDADANA, POLITICA, ELECCIONES, PROGRAMAS POLITICOS, DIFERENCIAS DE LEER EN MOVILES O PDF, COMENTAR PROGRAMAS, DESARROLLO COLABORATIVO (ECHAR UNA MANO O NECESIDAD DE AYUDA), PRUEBA DE CONCEPTO EXTENSIBLE A OTROS TEXTOS, APLICACION REAL, DESARROLLO DESDE CERO. 

APORTACION DE SOFTWARE LIBRE A LA COMUNIDAD PARA PROMOVER UNA PLATAFORMA QUE CUALQUIERA PUEDE USAR Y EN ANDROID EN LA QUE SOLO HABIA SOLUCIONES PRIVATIVAS.

PERDIDA DE INTERES POR INCOMPRENSION DE TEMAS TECNICOS, NO DAR POR HECHO QEUE TODO EL MUNDO ENTIENDE DE TODO.

MOVIL DISPOSITIVO ADECUADO PARA LEER PROGRAMAS, ESCRIBIR PROPUESTAS -> RAPIDAS, BAJO PLATAFORMA MAS COMODA, TECLADO. POSTURA NO ADECUADA PARA TRABAJAR (POCO TIEMPO, MAS ADECUADO TABLET). LIMITACION A UN SOLO DISPOSITIVO.

\section{Discusion de Resultados}
%\addcontentsline{toc}{chapter}{\numberline{Global Results}}

\section{Conclusiones}
%\addcontentsline{toc}{chapter}{\numberline{Conclusion}}

\section{Trabajo Futuro}

\subsection{Mejoras}

REALIZAR PARA OTRAS PLATAFORMAS -> WEB, NUEVAS FUNCIONALIDADES -> RECOMENDADORES SOCIALES, INTEGRACION CON REDES SOCIALES y MENSAJERIA INSTANTANEA (COMPARTIR), GRUPOS, DAR MAS FLEXIBILIDAD A LA HORA DE CREAR Y CLASIFICAR CONTENIDOS, COMPARATIVAS HECHAS POR USUARIOS

CONSEGUIR Participacion DE comunidades, analizar sus uso antes, durante y despues de elecciones.


