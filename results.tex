\newpage
\thispagestyle{sectioned}
\chapter{Resultados, Conclusiones y Trabajo Futuro}

En este capitulo analizaremos y discutiremos los resultados en conjunto de todo el Trabajo de Fin de Grado, las conclusiones que se pueden sacar tras su desarrollo y algunas líneas de trabajo futuro para este proyecto.

Todo el código open-source desarrollado por nosotros está disponible en el GitHub de la organización del proyecto:

\url{https://github.com/Zorbel}

PROPUESTAS, PARTICIPACION CIUDADANA, POLITICA, ELECCIONES, PROGRAMAS POLITICOS, 

DIFERENCIAS DE LEER EN MOVILES O PDF, COMENTAR PROGRAMAS, DESARROLLO COLABORATIVO (ECHAR UNA MANO O NECESIDAD DE AYUDA), PRUEBA DE CONCEPTO EXTENSIBLE A OTROS TEXTOS, APLICACION REAL, DESARROLLO DESDE CERO. 

APORTACION DE SOFTWARE LIBRE A LA COMUNIDAD PARA PROMOVER UNA PLATAFORMA QUE CUALQUIERA PUEDE USAR Y EN ANDROID EN LA QUE SOLO HABIA SOLUCIONES PRIVATIVAS.

PERDIDA DE INTERES POR INCOMPRENSION DE TEMAS TECNICOS, NO DAR POR HECHO QEUE TODO EL MUNDO ENTIENDE DE TODO.

MOVIL DISPOSITIVO ADECUADO PARA LEER PROGRAMAS, ESCRIBIR PROPUESTAS -> RAPIDAS, BAJO PLATAFORMA MAS COMODA, TECLADO. POSTURA NO ADECUADA PARA TRABAJAR (POCO TIEMPO, MAS ADECUADO TABLET). LIMITACION A UN SOLO DISPOSITIVO.

\section{Discusión de Resultados}

Inicialmente este proyecto nació para migrar la tecnología de colaboración en tiempo real de Apache Wave, presente en el proyecto SwellRT para plataformas web, a dispositivos móviles Android para posteriormente desarrollar una app que hiciera uso de esta tecnología. No habíamos trabajado nunca con tecnologías en tiempo real, así que tuvimos que realizar una pequeña investigación previa de los protocolos implicados (XMPP y WebSockets) y del funcionanmiento de Wave para ponernos al día. Además trabajamos junto a uno de los desarrolladores de SwellRT, que nos orientó a la hora de ''navegar'' por el código de la versión web para identificar las incompatibilidades con Android y subsanarlas. En poco tiempo tuvimos lista una versión simple del cliente capaz de conectarse al servidor Wave desde Android. Se trata de un desarrollo en código libre, por lo que realizamos una aportación a a un proyecto que cualquiera podría utilizar en su propio desarrollo Android. El resultado está deisponible en el GitHub de SwellRT:

\url{https://github.com/P2Pvalue/swellrt/tree/master/android}

Tocaba entonces embarcarse en la el desarrollo de la idea de alguna aplicación que hiciera uso de la tecnología de SwellRT. Se nos dejó total libertad para empezar este proyecto de aplicación desde cero y decidimos que si haciamos algo tenía que tener utilidad en el mundo real, más allá del alcance de este Trabajo de Fin de Grado. A continuación discutimos el por qué de DemoCritics y la utilidad de los resultados obtenidos. 


\subsection{El porqué de DemoCritics}

El desarrollo de herramientas que permitan explorar nuevas formas de colaborar, participar y expresar opiniones, o mejorar las herramientas ya existentes, siempre es un reto. Esto es así dada la gran cantidad de soluciones disponibles en la actualidad y ligadas sobre todo a la expansión que vivimos en los últimos años del uso para estos propósitos de Internet, que se ve reflejada en gran medida en las Redes Sociales. Cualquiera con conexión a la red tiene acceso a grandes cantidades de información que puede compartir y sobre la que puede opinar, ya no solo desde un ordenador sino también desde los ''ordenadores de bolsillo'' llamados smartphones que llevamos con nosotros prácticamente a todas partes.

Más específicamente, si nos centramos en el ámbito de la participación política y ciudadana, podemos detectar la reciente proliferación de herramientas que proporcionan métodos para organizarse, expresar opiniones, ponerse de acuerdo y construir nuevos proyectos políticos de forma más a menudo eficaz y eficiente que si se hiciera como tradicionalmente: ''cara a cara''. No hay más que ver cómo las nuevas formaciones y movimientos sociales, que tan a la orden del día estan con las recientes elecciones, hacen cada vez más uso de las nuevas tecnologías (como las descritas en el Estado del Arte de este documento) para organizarse y darse a conocer. 

Pero no solo los nuevos movimientos lo hacen. Los movimientos tradicionales (como los partidos políticos) son también conscientes de la importancia de estar presente en la red y adaptarse a las nuevas formas de comunicación. Así, encontramos que su presencia en Redes Sociales y la existencia de aplicaciones móviles para publicitar su mensaje son cada vez mayores. Sin embargo,   ¿No se supone que su principal mensaje está contenido en su programa electoral? ¿Dónde puedo ver ese mensaje? La práctica más comun hoy en día es la de colgar dicho programa electoral en su página web en forma de ''mega-documentos'' en PDF de gran extensión (los hay de hasta 200 páginas). Resultado: poca gente sabe de su existencia en las webs y los que lo saben rara vez se lo leen debido a su gran tamaño.

Tomando todo lo anterior en consideración hemos desarrollado \textbf{DemoCritics}, una plataforma para dispositivos móviles Android cuyo objetivo es \textbf{juntar en un único sitio el mensaje de la política tradicional (visto en forma de su Programa Político) con el mensaje de los ciudadanos (visto en forma de Propuestas)}.

Tratándose de una idea de proyecto desarrollado desde cero por nosotros, quisimos utilizar una metodología que nos permitiera investigar y diseñar una aplicación que realmente satisficiera necesidades reales y actuales. No obstante, a ambos simplemente nos interesaba la política. Ninguno de nosotros tenía el bagaje político necesario para poder considerarnos capaces de afirmar que nuestras ideas iniciales eran realmente viables a la hora de crear una aplicación que utilizara el público en general. Para ello fueron muy importantes las dos entrevistas que tuvimos oportunidad de hacer a personas involucradas en la participación politica como fueron los chicos de Labodemo y Javier de la Cueva. De esta fase de investigación desechamos algunas ideas iniciales y extrajimos otras nuevas más interesantes, plasmándose algunas de ellas en el estado actual de DemoCritics. 

\subsection{Usando Wave/SwellRT en DemoCritics}

No nos podemos olvidar del ''leitmotiv'' inicial de este proyecto, utilizar las capacidades de colaboración en tiempo real de Wave/SwellRT en Android. Por ello lo primero que pensamos fue generar las propuestas ciudadanas de forma colaborativa en un texto editable por cualquier usuario en tiempo real. Sin embargo, tras la discusión en las entrevistas entendimos que esto podría ser algo caótico a la hora de generar una propuesta útil, ya que al tratarse de algo tan subjetivo y personal, alcanzar un consenso era complicado. 

En su lugar, identificamos un problema de las propuestas que actualmente se realizan en la red: es muy fácil proponer algo pero a menudo la gente no tiene (ni tampoco es obligatorio que los tenga) los conocimientos necesarios para elaborar una propuesta viable y factible de ser llevada a cabo en un contexto actual. Al final, ¿qué diferencia una propuesta de realizar un comentario? ¿simplemente su longitud, que a priori es más larga en la propuesta? Tras debatir esta cuestión con Javier de la Cueva dimos con una solución a este problema. \textbf{Si queremos que las propuestas sean algo serio y razonablemente viable, ¿por qué no hacer que las propuestas vayan más allá de una declaración de intenciones y tengan asociadas campos que expliquen cómo se llevarían a cabo en la situación actual?} En este sentido identificamos dos elementos que nos parecieron clave para ello: que el usuario tras exponer su propuesta explicara ''¿Cómo la llevaría a cabo?'' (medidas a adoptar) y ''¿Cómo la financiaria?'' (necesidades de carácter monetario). Sin embargo un usuario no tiene por qué ser experto en estos temas ni tener muy claro cómo hacerlo, asi que ¿cual es la solución?.

Llegados a este punto retomamos la idea de usar Wave en las propuestas. Pero ahora no para redactar propuestas en su totalidad por parte de muchos usuarios, si no especificamente para \textbf{fomentar la colaboración entre usuarios}. Es muy común escuchar hablar de gente que quiere involucrarse en alguna tarea pero no sabe cómo puede echar una mano. Bien, pues con DemoCritics daríamos solución a las propuestas que el usuario no supiera cómo llevar a cabo poniendole en contacto con gente más versada en esos temas y que quisiera colaborar y echarle una mano. 

Así, nacieron las \textbf{propuestas colaborativas} (o ''abiertas''). Propuestas en las que un usuario al crearlas dejaba alguno de los dos campos (''¿Cómo la llevaría a cabo?'' y ''¿Cómo la financiaria?'') en blanco para permitir a otros usuarios que le echaran una mano mediante la edición colaborativa y en tiempo real de esa parte de la propuesta. De esta manera utilizamos Wave/SwellRT en un contexto más lógico que el de redactar la propuesta entera, pues al menos dichos campos estan sujetos a menos subjetividad que la declaración de intenciones de una propuesta más al uso de las que se encuentran en las plataformas actuales. Además, únicamente el usuario inicial de la propuesta es el único capaz de convertir la propuesta en ''definitiva'', cerrando la posibilidad de su edicion con Wave cuando estime que la comunidad de ''expertos'' ha podido ayudarle a proponer algo coherente y viable. 


\subsection{Incentivando el uso social de DemoCritics}

Por otro lado, hemos querido aprovechar el auge de las redes sociales para ''socializar'' la plataforma. De esta forma los usuarios pueden generar contenido e interés en el uso de la aplicación. Pueden dar su opinión mediante el uso de ''indicadores sociales'' similares a los ''Me gusta'' o No me gusta'' tan presentes en las redes sociales. También pueden realizar sus propios comentarios para generar debate y marcar contenido como favorito. Todo ello aplicado tanto a secciones de programas políticos como a propuestas hechas por otros usuarios. Esto, por otro lado, nos permite en cierta manera ''cuantificar'' el interés de las personas, lo cual siempre resulta útil para clasificar el contenido en forma de ''tops'' de lo más valorado, lo más comentado, etc. 

Aunque no solo es de utilidad para elaborar dichos filtros de visualización que atraigan la atención de los usuarios, sino que también los propios partidos políticos podrian beneficiarse de ello para identificar sus propuestas políticas más controvertidas o más valoradas. Al final se trata de una especie de ''red social'' aplicada al ámbito político en el que cualquiera de los actores puede intervenir y beneficiarse del contenido que genera la comunidad. 

También utilizamos un sistema de categorización tanto de secciones como de propuestas para llevar a cabo esa unión entre los programas politicos y las propuestas que tan necesaria nos parecía para fomentar tanto la lectura de secciones como la elaboración de propuestas. De hecho, esto atraería a usuarios que, lejos de querer leerse un programa político, lo que buscan es conocer el contenido de una determinada categoría de su interés. Aunque actualmente existen 6 categorías (Sanidad, Educación, Empleo, Vivienda, Impuestos y Cultura) la intención es que existan más y que incluso los propios usuarios interesados puedan crear sus propias categorías, tal y como se discutirá después en el trabajo a futuro. 

\subsection{Aspectos técnicos de DemoCritics}

Desde un punto de vista técnico lo primero que el desarrollo de DemoCritics nos ha exigido ha sido investigar un poco mejor el funcionamiento de la plataforma Android, tecnología para la que apenas habíamos desarrollado antes. Concretamente hemos tenido que indagar en aspectos tales como la conectividad con la red, el diseño y la interacción con la interfaz de usuario, la gestión de hilos y procesos, acceso a información del dispositivo móvil, etc. Afortunadamente no nos ha suspuesto mucho problema, pues la documentación oficial disponible por parte de Google es bastante amplia y explicativa. El resultado de esta aplicación se encuentra disponible para la comunidad en GitHub:

\url{https://github.com/Zorbel/appTFG}

Por otro lado, el diseño de la arquitectura del sistema nos llevó a plantearnos cómo desarrollar una plataforma que no nos restringiera su uso exclusivamente a Android. Queríamos que la interfaz de los datos fuera independiente de los datos en sí. En consecuencia, y tras evaluar distintas opciones, desarrollamos un Service REST que accediera a la Base de Datos. Este Service REST hace de intermediario entre las peticiones HTTP recibidas y el acceso a Base de Datos. De esta manera cualquier tipo de cliente con posibilidad de conexión a Internet puede acceder a dichos datos mediante peticiones HTTP. Los datos son entonces devueltos en formato JSON y tratados por el cliente.

Ambos teníamos conocimientos previos de Bases de Datos como para desarrollar la parte del alamacenamiento de los datos, pero ninguno de los dos conocía la tecnología detrás del Service REST. Sin embargo, si que teníamos algunos conocimientos de programación en PHP. Por tanto, dentro de la convicción de utilizar software libre, buscamos un framework que pudiera agilizar el desarrollo del Service REST en PHP y una plataforma online donde alojarlo. Como resultado utilizamos el framework open-source Laravel 5 para desarrollar el Service REST que alojamos en el servidor gratuito de RedHat OpenShift. 

El Service REST es accesible desde la siguiente URL:

\url{https://apptfg-servicerest.rhcloud.com/}

Asímismo, el código desarrollado para el Service REST está disponible en el GitHub del proyecto:

\url{https://github.com/Zorbel/ServiceRest}

FALTA WAVE

En el estado actual de la arquitectura del sistema detrás de DemoCritics, se ha comprobado que el cliente Android interactúa de forma correcta con el Service REST y la base de datos. Se pronpondran después en el trabajo futuro posibles mejoras a este sistema.

\section{Reparto de Trabajo entre los componentes del grupo} 

\section{Conclusiones}

\section{Trabajo Futuro}


REALIZAR PARA OTRAS PLATAFORMAS -> WEB, NUEVAS FUNCIONALIDADES -> RECOMENDADORES SOCIALES, INTEGRACION CON REDES SOCIALES y MENSAJERIA INSTANTANEA (COMPARTIR), GRUPOS, DAR MAS FLEXIBILIDAD A LA HORA DE CREAR Y CLASIFICAR CONTENIDOS, COMPARATIVAS HECHAS POR USUARIOS

CONSEGUIR Participacion DE comunidades, analizar sus uso antes, durante y despues de elecciones.


