\newpage
\thispagestyle{sectioned}

\appendix
\chapter{Listados de Código Fuente}

\section{Migración de Wave a Android: SwellRT}

\subsection{Esquema de conexión HTTP}\label{ssec:codeHTTP}

\lstset{language=Java, breaklines=true, autogobble=true, basicstyle=\ttfamily\footnotesize, commentstyle=\color{OliveGreen}, keywordstyle=\color{MidnightBlue}}
	  \begin{lstlisting}[frame=single]	  
import java.net.HttpURLConnection;

private void login(final String user, final String password, final Callback<String, String> callback) {
  
    //Construct the URL String urlStr with the server, user and password parameters
    URL url = new URL(urlStr); //String 
    HttpURLConnection connection = (HttpURLConnection) url.openConnection(); //Open the connection to the given URL 
    connection.setDoOutput(true); // allow the POST connection
    connection.setRequestProperty("Accept-Charset", CHARSET);
    connection.setRequestProperty("Content-Type", "application/x-www-form-urlencoded;charset=" + CHARSET);

    OutputStream out = connection.getOutputStream(); 
    out.write(queryStr.getBytes(CHARSET)); //Set the POST parameters

    if (connection.getResponseCode() != 200) {
	//ERROR during the connection
	connection.disconnect(); //Disconnect from the server.
    } else {
	//Continue with the login process (WebSocket)
	connection.disconnect(); //Disconnect from the server.
    }		      
}	    
	  \end{lstlisting}  
	  
\subsection{Esquema de conexión con AsyncTask}\label{ssec:codeAsynctask}	  
	  
	  \lstset{language=Java, breaklines=true, autogobble=true, basicstyle=\ttfamily\footnotesize, commentstyle=\color{OliveGreen}, keywordstyle=\color{MidnightBlue}}
	  \begin{lstlisting}[frame=single]
	  
private class LoginTask extends AsyncTask<String, Void, String> {
    
    @Override
    protected String doInBackground(String... params) { //method that executes on the new Thread without blocking the UI Thread
      login(params[0], params[1], params[2]); //Do the login 
      return sessionId //String needed for the WebSocket connection and based on the cookie received from the server.     
    }
   
   @Override
    protected void onPostExecute(String result) { //method that executes on the UI Thread once doInBackground() finishes its execution.    
      
      if (result != null) { 
        callback.onLogin(); //Notify the login success using the proper callback method
        
      } else { //The doInBackGround method has had a problem and the result of its execution was null
        callback.onError("Wave Login Error"); //Notify the login error using the proper callback method      
      }
    }
}    
	  \end{lstlisting} 
	  
\subsection{Esquema de uso de WAsync}\label{ssec:codewAsync}
	  	  
\lstset{language=Java, breaklines=true, autogobble=true, basicstyle=\ttfamily\footnotesize, commentstyle=\color{OliveGreen}, keywordstyle=\color{MidnightBlue}}
	  \begin{lstlisting}[frame=single]	  	  
	  
//Create the atmosphere client
AtmosphereClient client = ClientFactory.getDefault().newClient(AtmosphereClient.class);

//Configure client with URL
AtmosphereRequestBuilder requestBuilder = client.newRequestBuilder()
  .method(Request.METHOD.GET).trackMessageLength(true).uri(WaveSocketWAsync.this.urlBase)
  .transport(Request.TRANSPORT.WEBSOCKET)
  
//Create and configure socket
WaveSocketWAsync.this.socket = client.create(client.newOptionsBuilder().runtime(ahc).build())
  .on(Event.OPEN.name(), new Function<String>() { //Equivalent to GWT onOpen() method
    @Override
    public void on(String arg0) {
		// set the actions to do and call the proper callback function (callback.onConnect())
    }
    
  }).on(Event.CLOSE.name(), new Function<String>() { //Equivalent to GWT onClose() method
    @Override
    public void on(String arg0) {
    // set the actions to do and call the proper callback function (callback.onDisconnect())
    }
    
  }).on(Event.MESSAGE.name(), new Function<String>() {
    @Override
    public void on(String arg) { //Equivalent to GWT onMessage() method
    // set the actions to do and call the proper callback function (callback.onMessage())
    }
    
  }).on(new Function<Throwable>() {
    @Override
    public void on(Throwable t) {
		// catch possible exceptions
    }
  });
          
try {
// connect to the server
socket.open(requestBuilder.build());
} catch (IOException e) {
	// catch possible exceptions
}
      
//send a given message to the server, equivalent to GWT send(msg) method 
socket.fire(Data);
	\end{lstlisting}
	  
\section{Service REST (Laravel)}

\subsection{Esquema de Rutas Simple}\label{ssec:codeRoutesSimple}

\lstset{
  language        = php}
\begin{lstlisting}[frame=single]
// Posibles operaciones con el objeto "propuesta"
Route::get('/proposal', 'ProposalController@index');
  // Obtiene el listado de todas las propuestas
Route::get('/proposal/{id}', 'ProposalController@show');
  // Obtiene la prupuesta pasada por el campo {id}
Route::post('/proposal', 'ProposalController@create');
  // Crea una nueva propuesta en el servidor
Route::put('/proposal/{id}', 'ProposalController@update');
  // Actualiza la propuesta pasada por {id}
Route::delete('/proposal/{id}', 'ProposalController@destroy');
  // Borra la propuesta pasada por {id}
\end{lstlisting}

\subsection{Esquema de Rutas Agrupadas}\label{ssec:codeRoutesGroupe}

\lstset{
  language        = php}
\begin{lstlisting}[frame=single]
Route::group(['prefix' => 'proposal'], function()
{
    Route::get('/', 'ProposalController@index');
    Route::get('/{id}', 'ProposalController@show');
    Route::post('/', 'ProposalController@create');
    Route::put('/{id}', 'ProposalController@update');
    Route::delete('/{id}','ProposalController@destroy');
});
\end{lstlisting}

\subsection{Esquema de Rutas de Propuestas Agrupadas}\label{ssec:codeRoutesGroupeProposals}

\lstset{
  language        = php}
\begin{lstlisting}[frame=single]
Route::group(['prefix' => 'proposal'], function()
{
  Route::get('/', 'ProposalController@index');
  Route::post('/', 'ProposalController@create');
  Route::group(['prefix' => '/{id}'], function()
  {
    Route::get('/', 'ProposalController@show');
    Route::put('/', 'ProposalController@update');
    Route::delete('/','ProposalController@destroy');
  });
});
\end{lstlisting}

\subsection{Esquema de Controlador de Propuestas}\label{ssec:codeControllerProposals}

\lstset{
  language        = php}
\begin{lstlisting}[frame=single]
<?php namespace App\Http\Controllers;

use App\Http\Requests;
use App\Http\Controllers\Controller;
use DB;

class ProposalController extends Controller {
  /**
  * Display a listing of the resource.
  *
  * @return Response
  */
  public function index()
  {
    return DB::select('SELECT * FROM `proposal`');
  }
}
\end{lstlisting}

\section{Aplicación Android}

\subsection{Función de construcción del árbol de Programa Político}\label{ssec:codeProgramTree}

 \lstset{language=Java, breaklines=true, autogobble=true, basicstyle=\ttfamily\footnotesize, commentstyle=\color{OliveGreen}, keywordstyle=\color{MidnightBlue}}
	  \begin{lstlisting}[frame=single]	
protected int createIndex(Section parent, List<Section> JSONResult, int index) {

    // X = currentSection
    Section currentSection = JSONResult.get(index);

    if (parent.getlSections() == null) {
      parent.setlSections(new ArrayList<Section>());
    }
	    
    parent.addSubSection(currentSection);

    // c = nextIndex
    int nextIndex = index + 1;

    while ((nextIndex < JSONResult.size()) && (getLevel(JSONResult.get(nextIndex)) >= getLevel(currentSection))) {
      
      if (getLevel(JSONResult.get(nextIndex)) == getLevel(currentSection)) {
	currentSection = JSONResult.get(nextIndex);
	nextIndex++;
	parent.addSubSection(currentSection);
	
      } else if (getLevel(JSONResult.get(nextIndex)) > getLevel(currentSection))
	nextIndex = createIndex(currentSection, JSONResult, nextIndex);
    }
    
    return nextIndex;
}

protected int getLevel(Section sec) {
  
  int id_sec = sec.getmSection(), level;
  
  if (id_sec % 100 != 0) {
      level = 4;
      
  } else if (id_sec % 10000 != 0) {
      level = 3;
      
  } else if (id_sec % 1000000 != 0) {
      level = 2;
      
  } else
      level = 1;
      
  return level;
}	   
	  \end{lstlisting}
	  

\addcontentsline{toc}{chapter}{\numberline{}Bibliografía} %add bibliography to the index

\rhead{}
\renewcommand{\headrulewidth}{0pt}